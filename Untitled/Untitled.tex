% !TeX root = RJwrapper.tex
\title{Capitalized Title Here}
\author{by Author One, Author Two}

\maketitle

\abstract{%
The \pkg{regsem} package in R, an implementation of regularized
structural equation modeling {[}RegSEM; @jacobucci2016regularized{]},
was recently developed with the goal of incorporating various forms of
penalized likelihood estimation in a broad array of structural equations
models. The forms of regularization include both the \textit{ridge}
{[}@hoerl1970{]} and the least absolute shrinkage and selection operator
{[}\textbackslash{}textit\{lasso\}; @Tibshirani1996{]}, along with
sparser extensions. RegSEM is particularly useful for structural
equation models that have a small parameter to sample size ratio, as the
addition of penalties can reduce the complexity, thus reducing the bias
of the parameter estimates. The paper covers the algorithmic details and
an overview of the use of \pkg{regsem} with the application of both
factor analysis and latent growth curve models.
}

\subsection{Introduction}\label{introduction}

Introductory section which may include references in parentheses
\citep{R}, or cite a reference such as \citet{R} in the text.

\subsection{Section title in sentence
case}\label{section-title-in-sentence-case}

This section may contain a figure such as Figure \ref{figure:rlogo}.

\begin{figure}[htbp]
  \centering
  \includegraphics{Rlogo}
  \caption{The logo of R.}
  \label{figure:rlogo}
\end{figure}

\subsection{Another section}\label{another-section}

There will likely be several sections, perhaps including code snippets,
such as:

\begin{Schunk}
\begin{Sinput}
x <- 1:10
x
\end{Sinput}
\begin{Soutput}
#>  [1]  1  2  3  4  5  6  7  8  9 10
\end{Soutput}
\end{Schunk}

\subsection{Summary}\label{summary}

This file is only a basic article template. For full details of
\emph{The R Journal} style and information on how to prepare your
article for submission, see the
\href{https://journal.r-project.org/share/author-guide.pdf}{Instructions
for Authors}. \bibliography{RJreferences}

\address{%
Author One\\
Affiliation\\
line 1\\ line 2\\
}
\href{mailto:author1@work}{\nolinkurl{author1@work}}

\address{%
Author Two\\
Affiliation\\
line 1\\ line 2\\
}
\href{mailto:author2@work}{\nolinkurl{author2@work}}

